\section{Phase 1: Analyse}

\subsection{Beschreibung möglicher Anwendungen aus Business-Sicht}
\label{ref:businessSicht}

Um den längerfristigen Erfolg eines Unternehmen zu gewährleisten, ist es notwendig Kennzahlen und Faktoren für dieses zu definieren. 
Aus den gegebenen Zahlen lassen sich verschiedene Perspektiven ableiten. 
Die somit definierten Perspektiven sollten finanzielle Aspekte berücksichtigen, das Kaufverhalten des Kunden und die verbundenen Korellationen veranschaulichen, interne Probleme aufzeigen und die Entwicklung des Unternehmens allgemein beschreiben.\\
Im folgenden werden die einzelnen Perspektiven: Finanzperspektive, Kundenperspektive, Interne Perspektive und Entwicklungsperspektive vorgestellt.

\begin{table}[h]
\textbf{Finanzperspektive}
\begin{center}
  \begin{tabular}{ | l | r | }
    \hline
    \textbf{Ziel oder Fragestellung} & \textbf{Kennzahl} \\ \hline
    Kaufkraft der Bundesländer & Verkaufsanzahl \\  im Verhältnis zur Einwohnerzahl & (Quantität, zeitlich, Ort) \\ \hline
    Zshg. zwischen der Farbe, & Artikeleigenschaft,  \\ des Preises und verkaufte Einheiten & Preis, Verkaufsanzahl \\ \hline
    Preise von Produktgruppen im zshg. & Marktwert Rohstoff, \\ mit dem Markwert von Rohstoffen & Preis Produktgruppe \\ \hline
  \end{tabular}
\end{center}
\caption{Finanzperspektive}
\label{table:tableFinanz}
\end{table}

\begin{table}[h]
\textbf{Kundenperspektive}
\begin{center}
  \begin{tabular}{ | l | r | }
    \hline
    \textbf{Ziel oder Fragestellung} & \textbf{Kennzahl} \\ \hline
	Wer zahlt die Rechnung? & Eigenschaften von \\ Verwandte (Nachname gleich?)? & Kunde mit Bestellung \\ Welcher Altersgruppe gehört der zahlende an? & und zahlenden Kunden
    \\ \hline
    Zshg. von Körpergröße mit & Eigenschaften Artikel (Größe), \\ online/offline Einkäufen & -Bestellung, \\ und Retourenanzahl & und Anzahl der Retouren \\ \hline
  \end{tabular}
\end{center}
\caption{Kundenperspektive}
\label{table:tableKunde}
\end{table}

\begin{table}[h]
\textbf{Interne Perspektive}
\begin{center}
  \begin{tabular}{ | l | r | }
    \hline
    \textbf{Ziel oder Fragestellung} & \textbf{Kennzahl} \\ \hline
    Überblick über fehlerhafte Artikel & Fehlerhafte Artikel, Grund, Datum \\
    \hline
    Zeit Bestellung bis Auslieferung  & Durchschnittliche Verarbeitungsdauer \\ \hline
  \end{tabular}
\end{center}
\caption{Interne Perspektive}
\label{table:tableIntern}
\end{table}

\begin{table}[h]
\textbf{Entwicklungsperspektive}
\begin{center}
  \begin{tabular}{ | l | r | }
    \hline
    \textbf{Ziel oder Fragestellung} & \textbf{Kennzahl} \\ \hline
    Welches Bundesland benötigt & Verbrauch an Rohstoffen \\ wann mehr Rohstoffe &  pro Land und Zeit \\ und wie teuer wäre das? & \\ \hline
    Wie zahlen die Generationen? & Umsatz nach Alter pro Jahr \\ \hline
    Wer kauft online o. offline, & Eigenschaften Bestellung, \\ nach Altersgruppe und Stadt & Ort und Alter des Kunden \\ \hline 
  \end{tabular}
\end{center}
\caption{Entwicklungsperspektive}
\label{table:tableEntwicklung}
\end{table}

\subsection{Konzeptuelle Modellierung}
\label{ref:modellierung}

\subsection{Datenverarbeitungsanforderungen}
\label{ref:businessSicht}
