%allgemeine Formatangaben
\documentclass[
 a4paper, 										% Papierformat
 11pt,												% Schriftgröße
 ngerman, 										% für Umlaute, Silbentrennung etc.
 titlepage,										% es wird eine Titelseite verwendet
 oneside, 										% einseitiges Dokument
 captions=nooneline,					% einzeilige Gleitobjekttitel ohne Sonderbehandlung wie mehrzeilige Gleitobjekttitel behandeln
 numbers=noenddot,						% Überschriften-??Nummerierung ohne Punkt am Ende
 parskip=half,									% zwischen Absätzen wird eine halbe Zeile eingefügt
 ]{scrartcl}

% Anpassung an Landessprache
\usepackage[ngerman]{babel}	
\usepackage{acronym}
\usepackage[T1]{fontenc}	
\usepackage[utf8]{inputenc}	
\usepackage{textcomp} 																% Euro-Zeichen und andere
\usepackage[babel,german=quotes]{csquotes}						% Anführungszeichen
\RequirePackage[ngerman=ngerman-x-latest]{hyphsubst} 	% erweiterte Silbentrennung

% Befehle aus AMSTeX für mathematische Symbole z.B. \boldsymbol \mathbb
\usepackage{amsmath,amsfonts}

% Zeilenabstände und Seitenränder 
\usepackage{setspace}
\usepackage{geometry}

% Einbinden von JPG-Grafiken
\usepackage{graphicx}

% zum Umfließen von Bildern
% Verwendung unter http://de.wikibooks.org/wiki/LaTeX-Kompendium:_Baukastensystem#textumflossene_Bilder
\usepackage{floatflt}

% Verwendung von vordefinierten Farbnamen zur Colorierung
% Palette und Verwendung unter http://kitt.cl.uzh.ch/kitt/CLinZ.CH/src/Kurse/archiv/LaTeX-Kurs-Farben.pdf
\usepackage[usenames,dvipsnames]{color} 

% Tabellen
\usepackage{array}
\usepackage{longtable}

% einfache Grafiken im Code
% Einführung unter http://www.math.uni-rostock.de/~dittmer/bsp/pstricks-bsp.pdf
\usepackage{pstricks}

% Quellcodeansichten
\usepackage{verbatim}
\usepackage{moreverb} 											% für erweiterte Optionen der verbatim Umgebung
% Befehle und Beispiele unter http://www.ctex.org/documents/packages/verbatim/moreverb.pdf
\usepackage{listings}
\lstset{literate=
  {á}{{\'a}}1 {é}{{\'e}}1 {í}{{\'i}}1 {ó}{{\'o}}1 {ú}{{\'u}}1
  {Á}{{\'A}}1 {É}{{\'E}}1 {Í}{{\'I}}1 {Ó}{{\'O}}1 {Ú}{{\'U}}1
  {à}{{\`a}}1 {è}{{\`e}}1 {ì}{{\`i}}1 {ò}{{\`o}}1 {ù}{{\`u}}1
  {À}{{\`A}}1 {È}{{\'E}}1 {Ì}{{\`I}}1 {Ò}{{\`O}}1 {Ù}{{\`U}}1
  {ä}{{\"a}}1 {ë}{{\"e}}1 {ï}{{\"i}}1 {ö}{{\"o}}1 {ü}{{\"u}}1
  {Ä}{{\"A}}1 {Ë}{{\"E}}1 {Ï}{{\"I}}1 {Ö}{{\"O}}1 {Ü}{{\"U}}1
  {â}{{\^a}}1 {ê}{{\^e}}1 {î}{{\^i}}1 {ô}{{\^o}}1 {û}{{\^u}}1
  {Â}{{\^A}}1 {Ê}{{\^E}}1 {Î}{{\^I}}1 {Ô}{{\^O}}1 {Û}{{\^U}}1
  {œ}{{\oe}}1 {Œ}{{\OE}}1 {æ}{{\ae}}1 {Æ}{{\AE}}1 {ß}{{\ss}}1
  {ű}{{\H{u}}}1 {Ű}{{\H{U}}}1 {ő}{{\H{o}}}1 {Ő}{{\H{O}}}1
  {ç}{{\c c}}1 {Ç}{{\c C}}1 {ø}{{\o}}1 {å}{{\r a}}1 {Å}{{\r A}}1
  {€}{{\euro}}1 {£}{{\pounds}}1 {«}{{\guillemotleft}}1
  {»}{{\guillemotright}}1 {ñ}{{\~n}}1 {Ñ}{{\~N}}1 {¿}{{?`}}1
} 											% für angepasste Quellcodeansichten siehe
% Kurzeinführung unter http://blog.robert-kummer.de/2006/04/latex-quellcode-listing.html

\usepackage{pgfplots}
\usepackage{pgfplotstable}
\usepackage{filecontents}
\pgfplotsset{compat=1.9}

% verlinktes und Farblich angepasstes Inhaltsverzeichnis
\usepackage[pdftex,
colorlinks=true,
linkcolor=InterneLinkfarbe,
urlcolor=ExterneLinkfarbe]{hyperref}
\usepackage[all]{hypcap}

% URL verlinken, lange URLs umbrechen
\usepackage{url}

% sorgt dafür, dass Leerzeichen hinter parameterlosen Makros nicht als Makroendezeichen interpretiert werden
\usepackage{xspace}

% Beschriftungen für Abbildungen und Tabellen
\usepackage{caption}

% Entwicklerwarnmeldungen entfernen
\usepackage{scrhack}

\newcommand{\qq}[1]{\glqq{#1\grqq{}}} %Gänsefüßchen

\onehalfspacing 							% 1,5facher Zeilenabstand

\definecolor{InterneLinkfarbe}{rgb}{0.1,0.1,0.3} 	% Farbliche Absetzung von externen Links
\definecolor{ExterneLinkfarbe}{rgb}{0.1,0.1,0.7}	% Farbliche Absetzung von internen Links

% Einstellungen für Fußnoten:
\captionsetup{font=footnotesize,labelfont=sc,singlelinecheck=true,margin={5mm,5mm}}	
					
\title{Data Warehousing Projekt}
\subtitle{Analyse, Entwurf und Auswertung}

\author{Matthias Zober und Enrico Wüstenberg\vspace{5cm}}
\date{\today}
\begin{document}

\maketitle
\tableofcontents
\thispagestyle{empty}
\pagebreak
\setcounter{page}{1}

\section{Phase 1: Analyse}

\subsection{Beschreibung möglicher Anwendungen aus Business-Sicht}
\label{ref:businessSicht}
Im folgenden werden die einzelnen Perspektiven: Finanzperspektive, Kundenperspektive, Interne Perspektive und Entwicklungsperspektive vorgestellt.

\begin{table}[h]
\textbf{Finanzperspektive}
\begin{center}
  \begin{tabular}{ | l | r | }
    \hline
    \textbf{Ziel} & \textbf{Kennzahl} \\ \hline
    Einnahmen aller Artikel mit Bestelldaten & Umsatz Allgemein (Jahr, Monat) \\ \hline
    Rentabilität der Kunden & Anzahl Bestellungen pro Kunde \\
    \hline
    Welche Artikel verkaufen sich am besten? & Verkaufsanzahl (Quantität, zeitlich) \\
    \hline
    Überblick über ausstehende Einnahmen  & Ausstehende Einnahmen,  Liquidität  \\
    \hline
     Kreditverteilung der Kunden & Geburtsdatum, Kredit, Wohnort \\
    \hline
  \end{tabular}
\end{center}
\caption{Finanzperspektive}
\label{table:tableFinanz}
\end{table}

\begin{table}[h]
\textbf{Kundenperspektive}
\begin{center}
  \begin{tabular}{ | l | r | }
    \hline
    \textbf{Ziel} & \textbf{Kennzahl} \\ \hline
    Überblick über die Anzahl der Reklamationen & Reklamationsanzahl \\ \hline
    Anzahl der verkauften Artikel mit Bestelldatum & Bestseller, Jahreszeit \\
    \hline
    Artikel mit Retouren und Grund  & Kundenzufriedenheit \\ \hline
    Einfluss der Körpergröße auf Umsatz & Artikelgröße, Umsatz, Geschlecht \\
    \hline
    Wer zahlt die Rechnung? & Kunde mit Bestellung, zahlender Kunde \\
    \hline
  \end{tabular}
\end{center}
\caption{Kundenperspektive}
\label{table:tableKunde}
\end{table}

\begin{table}[h]
\textbf{Interne Perspektive}
\begin{center}
  \begin{tabular}{ | l | r | }
    \hline
    \textbf{Ziel} & \textbf{Kennzahl} \\ \hline
    Anzahl neuer Artikel pro Jahr & Neue Artikel, Datum\\ \hline
    Überblick über fehlerhafte Artikel & Fehlerhafte Artikel, Grund, Datum \\
    \hline
    Zeit Bestellung bis Auslieferung  & Durchschnittliche Verarbeitungsdauer \\ \hline
  \end{tabular}
\end{center}
\caption{Interne Perspektive}
\label{table:tableIntern}
\end{table}

\begin{table}[h]
\textbf{Entwicklungsperspektive}
\begin{center}
  \begin{tabular}{ | l | r | }
    \hline
    \textbf{Ziel} & \textbf{Kennzahl} \\ \hline
    Einkäufe Frau und Mann & Umsatz, Anzahl Artikel (Geschlecht, Alter) \\ \hline
    Kundenzuwachs nach Zeit & Neukunde, Lebensdauer Kunde \\
    \hline
    Wie zahlen die Generationen? & Umsatz nach Alter pro Jahr \\ \hline
   Vergleich offline/online Einkäufe & Alter, Geschlecht, online Kauf, offline Kauf \\ \hline
  \end{tabular}
\end{center}
\caption{Entwicklungsperspektive}
\label{table:tableEntwicklung}
\end{table}

\subsection{Konzeptuelle Modellierung}
\label{ref:modellierung}

\subsection{Datenverarbeitungsanforderungen}
\label{ref:businessSicht}

\section{Phase 2: Entwurf des Data Warehouse}

\subsection{Relationale Umsetzung eines MDM-Schemas}
\label{ref:mdmSchema}

\subsection{Optimierung der Data Cubes}
\label{ref:optimierung}

\end{document}